
\begin{figure*}[b]
\figline
 \caption{{\em Sample Grammar}.}
\singlespace
\begin{verbatim}
     element --> ID "(" expression_list ")"     // array reference
     element --> ID "(" expression_list ")"     // function call
\end{verbatim}
\doublespace
   \label{predicated}
\end{figure*}



In this section,
we overview terminology relating to name lookup
and discuss previous research that combines software
engineering technology to address the name lookup
problem for {\CPP}\cite{powertools}.
Much of the work on predicated LL(k) parsing,
and {\CPP} parser technology, has benefitted from
the initial research of James Roskind and
John Lilley\cite{lilley97,Roskind89}.

%\SubSection{The Roskind Grammar}
%\SubSection{The Work of Lilley}

\section{Predicated LL(k) parsing: ANTLR}

The ANTLR tool can be used to construct a predicated LL(k) parser
the output code is: {\CPP}, Java or Sather using the recursive descent
approach described in chapter \ref{back}.
Predicated LL(k) parsing is a version of LL(k) parsing
that exploits predicates to permit syntactic or semantic
information to direct the parse in a systematic way.
For example, the two grammar productions illustrated in
Figure \ref{predicated} look the same but have very
different semantic meanings. These two meanings can
be distinguished by consulting a symbol table to determine
the definition of {\em ID}.
If {\em ID} is defined as an array, then the first production
in Figure \ref{predicated} should be matched;
if {\em ID} is defined as a function name, then the second production
should be matched.

\begin{figure*}[t]
\singlespace  
\begin{verbatim}
     element() {
          if ( current_token == ID && isarray(current_token))
             match an array reference
          else if ( current_token == ID && isfunction(current_token))
             match a function call
     }
\end{verbatim}
   \caption{{\em Predicated LL(k) parsing}.}
\figline  
\doublespace
\label{semanticpred}
\end{figure*}

In a hand-built recursive descent parser, the user
might define function {\em element()}
as described in Figure \ref{semanticpred},
where {\sf isarray(current\_token)} is a function that
returns true if the ID is defined as an array and
where {\sf isfunction(current\_token)} returns
true if the ID is defined as a function.
This approach to parsing is called {\em predicated LL(k)}
parsing, {\sf pred-LL(k)}, because at least
one parsing decision was predicted using information
not available to an LL(k) parser.

The boolean functions {\sf isarray(current\_token)} and
{\sf isfunction(current\_token)} are semantic
predicates that can be incorporated into an ANTLR
parser using the notation {\sf ``$\{...\}?''$}.
For example, the hand-built parser for the grammar shown in
Figure \ref{semanticpred} might be defined in an
ANTLR grammar definition: 

\singlespace
\begin{verbatim}
  element --> {isarray(current_token)}? ID "(" expression_list ")"
  element --> {isfunction(current_token)}? ID "(" expression_list ")"
\end{verbatim}
\doublespace

The ANTLR tool also permits the user to incorporate 
syntactic predicates into the parse decision.
For example, the {\em Original grammar} illustrated
at the top of Figure \ref{toomany} can not be LL(k) since
the input string might contain {\em K + 1} A's.
This grammar can be augmented with a syntactic predicate,
as illustrated at the bottom of Figure \ref{toomany},
which indicates that to predict the first production,
zero or more A's must be seen followed by a B.~~
If this syntactic predicate fails, the second production
will be attempted by default.
The grammar in Figure \ref{toomany} also illustrates
that the ANTLR tool permits regular expressions;
in fact, the ANTLR tool permits 
Extended Backus-Naur Form, EBNF, to be used
to describe grammar productions.
The paper in reference \cite{parr} claims
that predicated LL(k) parsing can recognize a
larger class of languages than LL(k) parsing alone.

\begin{figure*}[t]
\singlespace
\begin{verbatim}
                 Original grammar:
                        a -> (A)* B
                        a -> (A)* C
                 ANTLR augmented grammar:
                        a : ( (A)* B) => (A)* B
                          | (A)* C
                          ;
\end{verbatim}
  \caption{{\em Syntactic predicate}.}
\doublespace
  \label{toomany}
\figline
\end{figure*}

\section{The Name Lookup Problem}

In this subsection, we define terms and introduce concepts
important for understanding the process of name lookup.
Name lookup is a necessary part of our parser front-end because it
allows us to find defintions for name uses and provide disambiguation for the grammar.
In the second subsection, we overview previous research that describes an approach for
constructing a symbol table for {\CPP}\cite{powertools}.

\section{Terminology and statement of the problem}

A {\em symbol table} is a data structure that stores information
about the types of the symbols and the names of the
symbols used in the program.  A symbol table for {\CPP} must capture and maintain
scope information.  Name lookup is complicated by the fact there are different rules
for name lookup in each scope defined by {\CPP}.  Some example scopes are: 
{\em namespace}, {\em class}, {\em enumeration} and {\em template}.  A further complication
related to name lookup is argument dependent lookup, or {\em Koenig} lookup.

\begin{figure*}[t]
 \centerline{\protect
\mbox{\psfig{figure=Figs/package.eps,height=.35\textheight}}}
  \caption{{\em Parser front-end design}.
}
   \label{fig:sysoverview}
\figline
\end{figure*}


\section{Previous Approach to the Name Lookup Problem}

Figure \ref{fig:sysoverview} summarizes the design of
a system to construct a parser front-end for ISO {\CPP}\cite{powertools}.
This contains two subsystems shown as folders in the figure.
The UML stereotype notation is used and each subsystem is denoted as $\ll${\sf subsystem}$\gg$.

The {\sf ProgramProcessor} subsystem includes a {\sf Scanner}, 
also know as a lexer, and a {\sf Parser}.  This system looks at the input
and directs symbol table construction and name lookup.  The ProgramProcessor 
follows an ordered process.  First, a {\sf NameOccurrence}
object is created.  Then a {\sf NameDeclaration} object is searched for in
the {\sf Symbol Table} subsystem.

The {\sf NameOccurrence} object encapsulates information
needed for lookup, including a representation of the name and context information related to
where the name occured.  The {\sf NameDeclaration} object holds information
including the name, type, and scope.  When this information is combined the system can
decide whether a name has been previously declared and whether the use of the name is allowed.



