

While the class diagrams of the previous section provide a general
framework within which name lookup takes place, we still have not
specified the actual rules used in deciding the sequence of scopes
that are searched, and which would be encoded in the {\tt lookup}
method for these scopes.

As might be expected with such a complex language as {\CPP},
there are many cases
to consider, and indeed, several examples are presented in the
{\CPP} standard.  In this section we present two specific cases, namespace
lookup and class-based lookup, which capture many of the fundamental
decisions involved in the lookup process.
For further information, the interested reader may consult
reference \cite{power99} for solutions to the name lookup
problem for virtually all of the examples listed in the {\CPP} standard.

\input{ns-program}

The examples presented below are based on some of those presented in
the {\CPP} standard.  These examples typically present a series of
declarations or definitions, and then some usages, and then describe
the name lookup for these usages.  We observe here that the standard UML
modeling techniques of object diagrams and sequence diagrams fit
smoothly onto this pattern, since the former are ideal for describing
the environment created by a declaration sequence, and the latter are
ideal for describing a particular invocation of the name lookup
procedure.


\begin{figure*}[t]
 \centerline{\protect
\mbox{\psfig{figure=../Figs/ns-obj.ps,height=.30\textheight}}}
  \caption{{\em Object diagram for the namespace example}.
This figure represents the two kinds of relationships between
namespaces: those established by a using directive and those
established by nesting.
}
   \label{ns-obj}
%\rule{6.5in}{0.3mm}\\
\end{figure*}


\section{Name Lookup for Namespaces}

Namespaces act as a modularization construct in {\CPP}, allowing the
programmer to partition the names used in a program to prevent them
from interfering with each other.  Thus, given some variable {\tt x}
declared in namespace {\tt A}, once outside namespace
{\tt A} we may refer to the variable using explicit qualification, as
in {\tt A::x}.

Namespaces may be nested inside each other, in which case name
occurrences inside the inner namespace may refer to those already
declared at the outer level without the need for qualification with
this process continuing recursively, eventually reaching the global
namespace where all namespaces are ultimately nested.

In addition to this textual relationship achieved through
qualification or nesting, we may establish a logical
relationship between namespaces by importing one into another with a
{\tt using} directive.  The declarations in a namespace that are
imported in this way are treated as though they were originally
declared in the importing namespace.




\begin{figure}[t]
 \centerline{\protect
\mbox{\psfig{figure=../Figs/ns-seq.ps,height=.30\textheight}}}
  \caption{{\em Sequence diagram for the namespace example}.
This figure captures the flow of control as messages are passed
through the namespace hierarchy. 
}
   \label{ns-seq}
%\rule{6.5in}{0.3mm}\\
\end{figure}



In Figure \ref{ns-program} we list a program segment to illustrate
name lookup for namespaces.
Here we declare five namespaces, with various using
relationships and one nesting relationship; we also
define a single function {\tt h()}.  

Our first step in representing this example is to construct an object
diagram, illustrated in Figure \ref{ns-obj},
showing how each scope instantiates a corresponding subclass of
our model's {\sf Scope} class, illustrated in Figure \ref{overview}
and discussed in Section \ref{approachoverview}.
The object diagram also represents the nesting and
import relationships between the namespaces.
For example, all namespaces are contained in an implicitly declared
global namespace; thus, namespaces {\sf Y}, {\sf AB},
{\sf B} and {\sf Z} of Figure \ref{ns-program} are implicitly
contained in an unnamed global namespace.
This nesting relationship is captured in Figure \ref{ns-obj}
by the unnamed class at the top of the figure
that is an instance of {\sf GlobalNamespace},
and classes {\sf Y}, {\sf AB},
{\sf B} and {\sf Z} related to
the unnamed instance of
{\sf GlobalNamespace} by the {\sf contained in} connector.
Similarly, namespaces {\sf A} and {\sf B} of Figure \ref{ns-program}
contain named using directives for {\sf Y} and {\sf Z}
respectively. This import relationship is captured
in Figure \ref{ns-obj} by the {\sf uses} connector between
instances of {\sf Y} and {\sf A}, and between {\sf Z} and {\sf B}.
Similar {\sf uses} relationships are shown in  Figure \ref{ns-obj}
between {\sf A} and {\sf AB} and between {\sf B} and {\sf AB}.

At the end of the program segment illustrated in Figure \ref{ns-program}
we have function {\tt h()} at global scope,
which contains a usage of the name {\tt f}, with an explicit
qualification by the namespace {\tt AB}.  To illustrate name lookup
for {\tt f}, we present a sequence diagram, illustrated in
Figure \ref{ns-seq}, indicating the
scopes that are searched and the order that they are
searched.  
The lookup proceeds as follows:

\begin{itemize}
\item Namespace {\tt AB} is first searched unsuccessfully; we must
next search its imported namespaces {\tt Z::A} and then {\tt B}.
\item The search of namespace {\tt Z::A} yields no declaration, and so
we search its imported namespace {\tt Y}.  Our search of {\tt Y} is
successful. 
\item Going back to {\tt AB} we are now directed to search namespace
{\tt B} where we also find a declaration, thus precluding the search
of its imported namespace {\tt Z}.
\end{itemize}

The search terminates, returning the two possible definitions of {\tt
f} to the program-processor, which will then proceed with the
subsequent stages of processing such as overload resolution.

Note that we do not consider the namespace {\tt Z} in which {\tt A} is
nested in this part of the search, nor do we search global namespace,
in accordance with the lookup rules for qualified names.  The presence
of a qualifier is made known to the {\sf Scope}'s {\tt lookup} method
through its parameter, the {\sf NameOccurrence} instance corresponding
to {\tt f}.

In the sequence diagram we are able to explicitly represent the roles
and responsibilities of each scope level.  For example, it is the
namespace {\tt AB} that causes a search of {\tt Z::A} and then {\tt B}
(in that order), and which is responsible for combining the results
from each of these searches.

