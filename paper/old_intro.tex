
In a world where the distributed computing environment is a cornerstone 
for millions of organizations, computer scientists continue to pursue new 
ways in which distributed applications can be developed rapidly and with ease.
This can be an ever challenging problem as organizations often have a need to 
utilize various computing platforms and programming languages.  Moreover, the 
desire to develop light-weight, modularized components using object oriented 
design principles has caused a software development paradigm shift.  The 
creation of more general purpose software components has become preferred over 
monolithic application development. In order to deal with these challenges, 
several new distributed computing technologies have been introduced.  One of 
these such technologies is the Common Object Request Broker Architecture 
(CORBA).  CORBA was developed by the Object Management Group (OMG).  The OMG 
is an open membership, not-for-profit consortium that produces and maintains 
computer industry specifications for interoperable enterprise applications.    
As of this writing, the OMG's membership roster includes about 800 of the 
largest companies in the computer industry.  One of the driving forces behind 
the creation of CORBA was the need for a high level of interoperability in a 
distributed environment.  In response to this, OMG introduced the Interface 
Definition Language (IDL).  By defining a standard interface for distributed 
objects in IDL, it is possible to interoperate with objects that may be 
implemented in different programming languages and that possibly exist on a 
remote host running a different operating system.  CORBA accomplishes this 
through the use of an Object Request Broker (ORB). 
The ORB is responsible for the actual communication between CORBA objects.  
In order to implement a CORBA object in a particular programming language, 
an ORB must exist that supports language bindings for that particular 
programming language. 
Many vendors readily offer ORBs for most of the popular programming
languages.  However, the more obscure programming languages have not been 
so fortunate when it comes to CORBA support.  Certain problem domains are 
more easily dealt with by using some of these more obscure languages.
This paper presents one approach for providing CORBA capabilties to 
programming languages that do not have ORBs available by implementing 
an Interface to an ORB with C language bindings, ORBit.

