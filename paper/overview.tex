In the next subsection, we define terms and introduce concepts
important for understanding the problem that we consider.
We then describe some of the difficulty involved in the
construction of
a parser front-end for C++, including the importance of
a solution to the name lookup problem.
In the second subsection, we overview our approach for
constructing a symbol table for C++ and the technique that we
use to perform name lookup.

\section{Terminology and statement of the problem}

Software developers need tools to facilitate
the design, analysis and testing of the system under development.
Many useful tools require a parser front-end to compute and
store information about the program, and to perform the analysis needed.
A {\em parser front-end} performs lexical and syntactic
analysis, constructs a symbol table, performs semantic
analysis and possibly generates an intermediate representation
of the program.
A {\em symbol table} is a data structure that stores information
about the types and names used in the program.

Many programming language constructs have an inherently recursive
structure that can be defined by context-free grammars, or CFGs\cite{aho86}.
Most of the constructs of languages such as Pascal and Ada
can be specified by CFGs, and parser front-ends for these languages
can base their recognition on syntactic considerations alone.
An exception to this easy-parse rule can be found in the language C,
where a declaration may not be easily distinguished from
an expression.  Consider the following code segment:

\begin{center}
\tt f (x);
\end{center}

\noindent
Intuitively, the above code segment appears to be an expression
involving an invocation of function {\tt f} with parameter {\tt x}.
However, if the context includes the following declaration: 

\begin{center}
\tt typedef int f;
\end{center}

\noindent
then the code segment is actually a declaration of {\tt x}
as an integer variable with redundant parentheses.
This {\em declaration/expression ambiguity}
not withstanding, parser front-ends for the
C language have not been difficult to construct.
However, a parser front-end for
the C++ language has proven elusive and 
the difficulties involved have been described in 
references \cite{bodin94,knapen99,lilley97,power00,reiss95,Roskind89}.
Currently, there is no parser front-end in the public domain
that can parse the language described in the ISO C++
standard\cite{C++standard98}. 
Many constructs in the C++ language cannot be recognized by
syntactic consideration alone;
these constructs
not only include the {\em typedef} declaration/expression
ambiguity of C, but C++ also includes context-dependent keywords
for {\em namespace}, {\em class}, {\em enumeration} and {\em template}
declarations\cite[Appendix A]{C++standard98}. 

To illustrate the declaration/expression ambiguity introduced 
by templates into C++,
consider that for a code segment such as {\sf a} $<$ {\sf b},
the name {\sf a} must be looked up to determine
whether the $<$ is the beginning of a template
argument list or a less-than 
operator \cite[\S 3.4.5/1]{C++standard98}.
Thus, the disambiguation of many C++ constructs
requires a solution to the name lookup problem.
The {\em name lookup} problem is defined as follows:
given the use of a name in the program, find the
corresponding declaration of that name.

The GNU Free Software Foundation
offers {\sf gcc},
a public domain compiler for the C++ language.
However, it is difficult to de-couple the parser front-end 
of {\sf gcc} from the back-end of the compiler.
Even if de-coupling were achieved, low-level access to
the internals of {\sf gcc} is not easily accomplished.
Furthermore, {\sf gcc} does not parse the language described
in the C++ standard.


\begin{figure*}[t]
 \centerline{\protect
\mbox{\psfig{figure=Figs/package.eps,height=.35\textheight}}}
  \caption{{\em System summary}.
This figure summarizes the design of our system
to construct a parser front-end for ISO C++.
The {\sf ProgramProcessor} subsystem is responsible
for initiating and directing symbol table construction
and name lookup by marshaling information about the name in a
{\sf NameOccurrence} object and directing the search for
a corresponding {\sf NameDeclaration} in
the {\sf Symbol Table} subsystem.
}
   \label{fig:sysoverview}
\figline
\end{figure*}


\section{Overview of our approach}

Figure \ref{fig:sysoverview} summarizes the design of
our system to construct a parser front-end for ISO C++.
The figure presents two subsystems, illustrated as
tabbed folders and designated by the $\ll${\sf subsystem}$\gg$ stereotype.
The {\sf ProgramProcessor} subsystem is shown on
the left side and the {\sf Symbol Table} subsystem is shown
on the right side of Figure \ref{fig:sysoverview}. 
The {\sf ProgramProcessor} and {\sf Symbol Table}
subsystems are elaborated in Sections \ref{processor}
and \ref{behavior} respectively.

The {\sf ProgramProcessor} subsystem includes a {\sf Scanner}
and {\sf Parser} and is responsible for initiating and
directing symbol table construction and name lookup.
This responsibility includes two phases: (1) assembling
the necessary information for creation of a {\sf NameOccurrence}
object, and (2) directing the search for a corresponding
{\sf NameDeclaration} object in the {\sf Symbol Table} subsystem.

The {\sf NameOccurrence} object encapsulates local information
relevant to the lookup, including the {\sf String}
representation of the name, a boolean to indicate
name qualification (by class or namespace), and
an enumeration, {\sf OccurSpecifier}, that
captures lexical information about the context
in which the name occurred. 
The {\sf NameDeclaration} object includes the
{\sf String} representation of the name,
an enumeration indicating the type of name,
and a pointer to the enclosing scope.
The {\sf NameOccurrence}
object is discussed further in the next section.



